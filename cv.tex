% !TeX program = xelatex

\documentclass[a4paper, 12pt]{article}

\usepackage{amsmath, mathtools}
\usepackage{blindtext}
\usepackage{fontspec}
\usepackage{layout}
\usepackage[
    verbose,
    paper=a4paper,
    % showframe,
    left=1in,
    right=1in,
    top=36pt,
    textheight=698pt,
    headheight=16pt,
    headsep=20pt, %top+headh+heads=72pt
    marginparwidth=0pt,
    marginparsep=0pt,
    includeheadfoot]{geometry}
\usepackage{fancyhdr}
\usepackage{lastpage}
\usepackage{titlesec}
\usepackage{xcolor}
\usepackage{enumitem}
\usepackage{array}
\usepackage[hyphens]{url}
\usepackage{hyperref}

\mathtoolsset{showonlyrefs}

% Set main font
\setmainfont{Times New Roman}

% Header and Footer
\pagestyle{fancyplain}
\fancyhf{}
\pagenumbering{arabic}
\fancyhead[C]{
    \textbf{Momentum MSCA Programme - Call 2}
}
\fancyfoot[R]{\scriptsize
    page \thepage \,out of\, \pageref{LastPage}}
\renewcommand{\headrulewidth}{0pt}
\renewcommand{\footrulewidth}{0pt}

\titleformat{\section}
    {\normalfont\bfseries}{\thesection.}{1em}{}

% Bibliography setup
\usepackage[
    style=ieee,
    backend=biber,
    url=false,
    doi=false,
    maxnames=4]{biblatex}
\addbibresource{refs.bib}
\defbibheading{bibliography}{\section*{References}}

\newcommand\mcl[1]{\multicolumn{2}{|l|}{#1}}

\def\CC{{C\nolinebreak[4]\hspace{-.05em}\raisebox{.2ex}{\scriptsize\bf ++}}}

%%%%%%%%%%%%%%%%%%%%%%%%%%%%%%%%%%%%%%%%%%%
\begin{document}

\begin{center}
\fbox{\parbox{0.98\textwidth}{\centering
\textbf{Narrative CV of the Applicant}
}}
\end{center}

\noindent\textbf{Personal details}
\vspace*{-5pt}
\begin{center}
    \renewcommand{\arraystretch}{1.6}
    \begin{tabular}{
        @{}|>{\raggedright}p{.2\textwidth-2\tabcolsep}
        |>{\raggedright\arraybackslash}p{.13\textwidth-2\tabcolsep}
        |>{\raggedright\arraybackslash}p{.25\textwidth-2\tabcolsep}
        |>{\raggedright\arraybackslash}p{.4\textwidth-2\tabcolsep}|
    @{}}
        \hline
        \mcl{Surname of the Applicant:} &
        \mcl{Ponce Vanegas} \\
        \hline
        \mcl{First name of the Applicant:} &
        \mcl{Felipe Eduardo} \\
        \hline
        \mcl{Researcher unique identifier:} &
        \mcl{ORCID 0000-0002-1049-9752} \\
        \hline
        \mcl{URL for web site:} &
        \mcl{\url{https://a-wandering-mathematician.netlify.app/}} \\
        \hline
        PhD awarded (year): & 2018 &
        Active Research Years (FTE)*: & 7 \\
        \hline
    \end{tabular}
\end{center}
\vspace*{-6pt}
{\small*Active Research Years (FTE)*: State the total number of years you have actively engaged in research since the award of your PhD (or equivalent), calculated in full-time equivalent (FTE).

\noindent Periods of career break, maternity, paternity or parental leave, long-term sick leave, national or military service, unemployment, non-research employment, or time spent outside main research activities should be deducted from the total Full-Time Equivalent (FTE) research period.}


\section*{Professional career in narrative form}

I completed my B.Sc. in Chemistry (April 2011) and my M.Sc. in mathematics (April 2014) at
the National University of Colombia (UNAL), Bogotá, Colombia.
The main topic of my Master's thesis was the mathematical theory of elastic shells, and,
as a side project, I simulated spatial patterns of oscillating chemical reactions in \CC.

During my work on elastic shells,
I was impressed by the effectiveness of harmonic analysis (a field of pure mathematics) for
proving deep inequalities.
Harmonic analysis stu\-dies, among other things, the theoretical foundations of signal decompositions
like the Fourier transform or the wavelet transform, which
are nowadays important tools for signal processing in engineering.

Thus, I decided to pursue a Ph.D. in mathematics at the UNAL
under the direction of Javier Ramos Maravall, who
I met at the Institute of Pure and Applied Mathematics in Brazil
where he was a distinguished postdoc in harmonic analysis; today,
he is a professor at the Autonomous University of Madrid, Spain.
I did a secondment (Jan. - Jun. 2017) at
the Institute of Mathematical Sciences, Madrid, Spain.

After finishing my Ph.D. --- and a brief period of unemployment --- I was hired in May 2019 by the Basque Center for Applied Mathematics (BCAM) in Bilbao, Basque Country, Spain.
Here, I conducted research in pure mathematics, mainly
harmonic analysis and applications to other fields of mathematics
like inverse problems or partial differential equations.

By the beginning of 2022 I started feeling dissatisfied with the very abstract nature of my research, and
I began to look on my own for industrial problems where my skills could be relevant;
around this time (March 2022) I started programming again.

By the Fall of 2022, Luis Vega, a very well-known researcher in harmonic analysis at BCAM,
introduced me to Norberto López de Lacalle,
the scientific director of the Aeronautics Advanced Manufacturing Centre (CFAA).
This was the beginning of a new era for me, and
I was introduced to CNC machining by Michael Barto\v{n}, a researcher at BCAM who collaborates with CFAA.
We soon started working in a project on the smoothness of
surfaces manufactured by flank milling in 5-axis CNC machines which was published.

After a successful takeoff towards new lands,
I moved my office to CFAA. However,
I had a tough time there because I did not find
where my research could fit in, and
it was my first experience surrounded by engineers, who
talked a different language than mine.
Eventually, I had to adjust to their problems and
get into fields of mathematics I was not an expert on, but
which were relevant for my colleagues at CFAA.

Among the many subjects I explored,
I worked on Delay Differential Equations (DDE), which
appear in the study of chatter in machining by metal removal.
I had the fortune to discuss the topic with Jokin Muñoa (IDEKO),
who introduced me to Zoltán Dombóvári.
He posed a question to me about stability of tool dynamics where parameters changed slowly,
which was the beginning of a fruitful collaboration.
Eventually, he invited me to work with him and his group at
the Budapest University of Technology and Economics (BME),
where I spent the Fall of 2023 (Oct. - Nov.).

I also started another important project
when a colleague at CFAA proposed that
I colla\-bo\-rated with them in a problem suggested by ITP Areo, a very important company in Spain.
This forced me to learn new skills and allowed me
to gain experience in data management.
In particular,
I found that Bayesian statistics was the right point of view for the project.

At present,
I continue working at BCAM within a collaboration agreement with CFAA.
Motivated by a project at CFAA on modelling of turning,
I got interested in Experimental Modal Analysis,
where I am revisiting several aspects using recent mathematical tools, and
I expect to publish these results together with an open Python package this year.

My choice of departing from pure mathematics in favor of more applied topics was for me, without question, the right one.
Today, I feel motivated and eager to pursue further research on
problems with industrial relevance.


\section*{Research achievements}

As I mentioned above,
my career is roughly divided into my time as a pure mathematician (before mid-2022), and as an applied mathematician (from mid-2022 to present).
Although my work in pure mathematics gave me a solid theoretical background,
I will focus on my current role in applied mathematics.

In an article with Michal Bizarri and Michael Barto\v{n},
we studied the smoothness of surfaces generated by 5-axis CNC machining.
We wanted to understand the influence of the smoothness
of the tool's profile and its motion path on
the smoothness of the swept surface.
We found several theoretical conditions for the generation of continuous and differentiable surfaces,
which are desirable properties in the final product.
For example,
if the tool has a corner, then, in general, the swept surface is discontinuous, but
we found sufficient and necessary conditions for the motion to generate a continuous surface.

M. Barto\v{n} proposed the initial question, and
I was able to answer it. Then,
I contributed in the proof of all subsequent theorems, and
I also developed a Matlab code to visualize the surfaces, which is publicly available.
This shows my capacity for linking abstract concepts with applied problems, and
its implementation in a program.

In an article with András Bartfái and Zoltán Dombóvári,
we develop a method to solve Ordinary Differential Equations (ODEs)
when parameters slowly transverse a bifurcation.
This problem was motivated by the interest of Z. Dombóvári
in understanding the dynamics of robots and other
structures where parameters cannot be assumed to be constant.
However, he and his team faced huge difficulties
because standard solvers, like Runge--Kutta schemes,
fail drama\-ti\-ca\-lly when there is a bifurcation.

Z. Dombóvári and A. Bartfái had a method to solve a specific ODE,
but the method did not have a rigorous justification but a heuristic support.
I helped them by providing a rigorous proof so that
they can trust the solution they obtain, and
I also contributed generalizing the method to a broader family of ODEs.
This output shows that I have already conducted research in problems
closely related to the proposed project, and
that I understand the nuances of it.
Again, the Python code was made publicly available.

In a project with CFAA researchers, and
promoted by ITP Aero,
I collaborated in an industrially relevant problem of tool wear estimation in drilling of hard-to-cut materials.
I curated the experimental data, and
I also participated in the planning of the experiments at different stages.

I proposed a general methodological framework for monitoring of tool wear processes, which
is based on state-space models.
The tool wear, which follows a nonlinear dynamics,
is the latent or hidden variable, and
the value recorded by external sensors is the observed variable.
The model also takes into account the tool-to-tool variability,
which is usually neglected, although
we found it has an important impact in the performance of the process.
The analysis of the data was done using Bayesian inference, which
is the right technique for the statistical analysis of experimental data,
but usually sidelined in favor of methods that are conceptually and computationally less demanding.

The conclusions of this work were important for ITP Aero.
This project shows my ability to translate problems in engineering
into abstract frameworks that allow a rigorous analysis and generalization.
Also, it shows that I already have experience with Bayesian methods.
The data and the code are publicly available.


\section*{Peer recognition}

I was awarded the Yu Takeuchi prize to the best Ph.D. thesis by the  Colombian Academy of Exact, Physical and Natural Sciences (3 Dec., 2018).
Working at BCAM, I was awarded the Juan de la Cierva Scholarship 2019,
a prestigious grant in Spain that
secured me a two-year extension (1 Sep., 2021 - 30 Jun., 2023).


\section*{Other relevant information and contributions to the research community}

Presently, I participate as a BCAM's team member in a large academic-industry national project titled 
\emph{Federated process management architecture in discrete event
manufacturing systems} (SARA), funded
by the State Research Agency (AEI) and
the Centre for the Development of Industrial Technology (CDTI).
The industrial consortium is led by Fagor Automation, and
the academic consortium by the University of the Basque Country (EHU) through the CFAA.
My tasks include monitoring and modelling of manufacturing processes.

Previously, I was funded by the Basque Government through the IKUR program
for a project on the application of modern mathematical methods to Advanced Manufacturing (30 Jun., 2023 - 30 Jun., 2025).

I participated in a European Study Group with Industry (ESGI 188), Bilbao, Spain (26-30 May 2025).
I was the scientific supervisor of Danobat Group's challenge
on machine tool vibrations and stability.

Supervision:
I co-directed, with Luis Vega, the Bachelor's thesis of Markel Irastorza at the EHU on
the stability of DDEs (defense date: 7 July, 2024).
Currently, I am the supervisor of Reinaldo Espina-Navarro,
a research technician at BCAM (July 2025 - present), and
currently I am supervising Andoni Uyarra, a Master student in mathematics from the EHU.

Other activities: co-organizer of the \emph{Bilbao Analysis and PDE} seminar (Feb.-Aug. 2022).
Lecturer of a BCAM course on \emph{Git for mathematicians in the writing of texts} (March 2022); see also \url{https://gitlab.bcamath.org/fponce/git-for-mathematicians/}.
Lecturer of a BCAM course on \emph{Diophantine Approximation through the Mass Transference Principle} (21-25 May, 2022).


\section*{Publications and Other Research Outputs}


\begin{itemize}[wide, labelindent=0pt]
    \setlength\itemsep{-2pt}
    \item \textbf{Full reference:}
    Ponce-Vanegas, F., Artabe, A., Polvorosa, R., Espina-Navarro, R., Fernández, A. and López de Lacalle, N.
    (2026).
    A Bayesian state-space model for tool wear estimation in drilling of Inconel~718. Preprint.
    \url{}
    \item \textbf{Role and contribution:}
    First author, conceptualization, data curation, formal analysis, software, supervision of Espina-Navarro, and manuscript writing.
    \item \textbf{Evidence of impact:}
    Preliminary tests. Research promoted by ITP Aero, a Spanish aero engine manufacturer, for potential use in their production plant.
    \item \textbf{Accessibility:}
    Submitted version: \url{http://hdl.handle.net/20.500.11824/2149}

    \noindent Curated dataset: \url{https://doi.org/10.82518/TTACDO}
\end{itemize}


\begin{itemize}[wide, labelindent=0pt]
    \setlength\itemsep{-2pt}
    \item \textbf{Full reference:}
    Irastorza, M. and Ponce-Vanegas, F.
    (2026).
    The Tustin Method for approximating eigenvalues of delay systems.
    \textit{J. Comput. Appl. Math.},
    472, 116772.
    \url{https://doi.org/10.1016/j.cam.2025.116772}
    \item \textbf{Role and contribution:}
    Conceptualization, methodology, software, supervision, and ma\-nus\-cript writing.
    \item \textbf{Evidence of impact:}
    This is the Bachelor's thesis in mathematics of Markel Irastorza.
    \item \textbf{Accessibility:}
    Submitted version:
    \url{http://hdl.handle.net/20.500.11824/1841}.
\end{itemize}


\begin{itemize}[wide, labelindent=0pt]
    \setlength\itemsep{-2pt}
    \item \textbf{Full reference:}
    Ponce-Vanegas, F., Bartfai, A., and Dombovari, Z.
    (2025).
    Semi-analytical Estimation for the Escape of Solutions of Linear Differential Equations with Slowly Varying Coefficients.
    \textit{SIAM J. Appl. Math.},
    85(4), 1519-1549.
    \url{https://doi.org/10.1137/24M1685481}
    \item \textbf{Role and contribution:}
    First author, conceptualization, methodology, software, and ma\-nus\-cript writing.
    \item \textbf{Accessibility:}
    Submitted version:
    \url{http://hdl.handle.net/20.500.11824/1830}
\end{itemize}


\begin{itemize}[wide, labelindent=0pt]
    \setlength\itemsep{-2pt}
    \item \textbf{Full reference:}
    Ponce-Vanegas, F., Bizzarri, M., and Barton, M.
    (2023).
    On and continuity of envelopes of rotational solids and its application to 5-axis CNC machining.
    \textit{CAGD},
    107, 102245.
    \url{https://doi.org/10.1016/j.cagd.2023.102245}
    \item \textbf{Role and contribution:}
    First author, conceptualization, methodology, software, and ma\-nus\-cript writing.
    \item \textbf{Accessibility:}
    Submitted version:
    \url{http://hdl.handle.net/20.500.11824/1638}
\end{itemize}


\begin{itemize}[wide, labelindent=0pt]
    \setlength\itemsep{-2pt}
    \item \textbf{Full reference:}
    Eceizabarrena, D. and Ponce-Vanegas, F.
    (2022).
    Pointwise convergence over fractals for dispersive equations with homogeneous symbol.
    \textit{J. Math. Anal. Appl.},
    515, 126385.
    \url{https://doi.org/10.1016/j.jmaa.2022.126385}
    \item \textbf{Role and contribution:}
    Conceptualization, methodology, and ma\-nus\-cript writing.
    Equal contribution with D. Eceizabarrena.
    \item \textbf{Accessibility:}
    Submitted version:
    \url{https://doi.org/10.48550/arXiv.2108.10339}
\end{itemize}


\begin{itemize}[wide, labelindent=0pt]
    \setlength\itemsep{-2pt}
    \item \textbf{Full reference:}
    Ponce-Vanegas, F.
    (2020).
    A Trilinear Restriction Estimate for the Hyperbolic Paraboloid with Sharp Dependence on Transversality.
    \textit{Int. Math. Res. Not.},
    2020(18), 5723--5753.
    \url{https://doi.org/10.1093/imrn/rny178}
    \item \textbf{Role and contribution:}
    Conceptualization, methodology, and ma\-nus\-cript writing.
    \item \textbf{Evidence of impact:}
    This was part of my Ph.D. thesis for which I was awarded the Yu Takeuchi prize; see Peer recognition.
\end{itemize}


\noindent Please provide one or more permanent links where your publication record is available:
\begin{center}
    \renewcommand{\arraystretch}{1.2}
    \begin{tabular}{
        @{}|>{\raggedright}p{.25\textwidth-2\tabcolsep}
        |>{\raggedright\arraybackslash}p{.74\textwidth-2\tabcolsep}|
    @{}}
        \hline
        \textbf{Source} & \textbf{Link} \\ \hline
        ORCID & \url{https://orcid.org/0000-0002-1049-9752} \\ \hline
        Personal webpage & \url{https://a-wandering-mathematician.netlify.app/about} \\ \hline
    \end{tabular}
\end{center}

\end{document}