% !TeX program = xelatex

\documentclass[a4paper, 12pt]{article}

\usepackage{amsmath, mathtools}
\usepackage{blindtext}
\usepackage{fontspec}
\usepackage{layout}
\usepackage[
    verbose,
    paper=a4paper,
    % showframe,
    left=1in,
    right=1in,
    top=36pt,
    textheight=698pt,
    headheight=16pt,
    headsep=20pt, %top+headh+heads=72pt
    marginparwidth=0pt,
    marginparsep=0pt,
    includeheadfoot]{geometry}
\usepackage{fancyhdr}
\usepackage{lastpage}
\usepackage{titlesec}
\usepackage{xcolor}
\usepackage{enumitem}
\usepackage{array}
\usepackage[hyphens]{url}
\usepackage{hyperref}

\mathtoolsset{showonlyrefs}

% Set main font
\setmainfont{Times New Roman}

% Header and Footer
\pagestyle{fancyplain}
\fancyhf{}
\pagenumbering{arabic}
\fancyhead[C]{
    \textbf{Momentum MSCA Programme - Call 2}
}
\fancyfoot[R]{\scriptsize
    page \thepage \,out of\, \pageref{LastPage}}
\renewcommand{\headrulewidth}{0pt}
\renewcommand{\footrulewidth}{0pt}

\titleformat{\section}
    {\normalfont\bfseries}{\thesection.}{1em}{}

% Bibliography setup
\usepackage[
    style=ieee,
    backend=biber,
    url=false,
    doi=false,
    maxnames=4]{biblatex}
\addbibresource{refs.bib}
\defbibheading{bibliography}{\section*{References}}

\newcommand\mcl[1]{\multicolumn{2}{|l|}{#1}}

\def\CC{{C\nolinebreak[4]\hspace{-.05em}\raisebox{.2ex}{\scriptsize\bf ++}}}

%%%%%%%%%%%%%%%%%%%%%%%%%%%%%%%%%%%%%%%%%%%
\begin{document}

\begin{center}
\fbox{\parbox{0.98\textwidth}{\centering
\textbf{Narrative CV of the Applicant}
}}
\end{center}

\noindent\textbf{Personal details}
\vspace*{-5pt}
\begin{center}
    \renewcommand{\arraystretch}{1.6}
    \begin{tabular}{
        @{}|>{\raggedright}p{.2\textwidth-2\tabcolsep}
        |>{\raggedright\arraybackslash}p{.13\textwidth-2\tabcolsep}
        |>{\raggedright\arraybackslash}p{.25\textwidth-2\tabcolsep}
        |>{\raggedright\arraybackslash}p{.4\textwidth-2\tabcolsep}|
    @{}}
        \hline
        \mcl{Surname of the Applicant:} &
        \mcl{Ponce Vanegas} \\
        \hline
        \mcl{First name of the Applicant:} &
        \mcl{Felipe Eduardo} \\
        \hline
        \mcl{Researcher unique identifier:} &
        \mcl{ORCID 0000-0002-1049-9752} \\
        \hline
        \mcl{URL for web site:} &
        \mcl{\url{https://a-wandering-mathematician.netlify.app/}} \\
        \hline
        PhD awarded (year): & 2018 &
        Active Research Years (FTE)*: & 7 \\
        \hline
    \end{tabular}
\end{center}
\vspace*{-6pt}
{\small*Active Research Years (FTE)*: State the total number of years you have actively engaged in research since the award of your PhD (or equivalent), calculated in full-time equivalent (FTE).

\noindent Periods of career break, maternity, paternity or parental leave, long-term sick leave, national or military service, unemployment, non-research employment, or time spent outside main research activities should be deducted from the total Full-Time Equivalent (FTE) research period.}


\section*{Professional career in narrative form}

I completed my B.Sc. in Chemistry and my M.Sc. in mathematics at
the National University of Colombia, Bogotá, Colombia.
The main topic of my master was the mathematical theory of elastic shells, and
as a side project I worked on simulations of spatial patterns of oscillating chemical reactions using \CC.

Then, I pursued my Ph.D. in mathematics at the same university
under the direction of Javier Ramos Maravall (today at the Autonomous University of Madrid), and
I did a secondment (Jan. - Jun., 2017) at
the Institute of Mathematical Sciences, Madrid, Spain.
The topic was harmonic analysis, an area of pure mathematics.
Harmonic analysis studies, among other things, the theoretical foundations of signal decompositions
like the Fourier transform or the Wavelet transform, which
are today important tools for signal processing in engineering.

After my Ph.D. --- and a brief period of unemployment --- I was hired in May 2019 by the Basque Center of Applied Mathematics in Bilbao, Basque Country, Spain.
I conducted research in pure mathematics, mainly
Harmonic Analysis and applications in other fields of mathematics
like Inverse Problems or Partial Differential Equations.

By the beginning of 2022 I started feeling dissatisfied with the abstract nature of my research, and
I began to look by myself for industrial problems where my skills could be relevant.
Around this time (March 2022) I started programming again, and
I even taught a course on Git for mathematicians in the writing of papers; see \url{https://gitlab.bcamath.org/fponce/git-for-mathematicians/}

By Fall 2022, Luis Vega (BCAM), a very well-known researcher in Harmonic Analysis,
introduced me to Norberto López de Lacalle,
the scientific director of the Aeronautics Advanced Manufacturing Centre (CFAA).
This was the beginning of a new era for me, and,
as an introduction to the field,
I collaborated with Michael Barto\v{n} in a project on the smoothness of
surfaces manufactured by flank milling in 5-axis CNC machines.

After a successful take-off in this new field,
I changed my office to CFAA, but
I had a tough time there because I did not find
where my research could fit in, and
it was my first experience surrounded by engineers, which
talked a different language than mine.
Eventually, I had to adjust to their problems and
get into fields of mathematics I was not expert on, but
relevant for my colleagues at CFAA.

Among the many subjects I explored,
I worked on Delay Differential Equations (DDE), which
appear in the study of chatter in machining by metal removal.
I had the fortune to discuss the topic with Jokin Muñoa (IDEKO),
who introduced me to Zoltán Dombóvári.
He posed me a question about stability of tool dynamics when
parameters change slowly,
which was the beginning of a fruitful collaboration, and
he invited me to work with him and his group at
the Budapest University of Technology and Economics (BME),
where I spent the Fall 2023 (Oct. - Nov.).

In another work on DDEs,
I co-directed, with Luis Vega, the B.Sc. thesis of Markel Irastorza at the University of the Basque Country (EHU).
We developed a new algorithm for computing the characteristic roots of 


\section*{Research achievements}

\section*{Peer recognition}

\section*{Other relevant information and contributions to the research community}

\section*{Publications and Other Research Outputs}


\begin{itemize}[wide, labelindent=0pt]
    \setlength\itemsep{-2pt}
    \item \textbf{Full reference:}
    Ponce-Vanegas, F., Artabe, A., Polvorosa, R., Espina-Navarro, R., Fernández, A. and López de Lacalle, N.
    (2026).
    A Bayesian state-space model for tool wear estimation in drilling of Inconel~718. Preprint.
    \url{}
    \item \textbf{Role and contribution:}
    First author, conceptualization, data curation, formal analysis, software, and manuscript writing.
    \item \textbf{Evidence of impact:}
    Preliminary tests. Research promoted by ITP Aero, a Spanish aero engine manufacturer, for potential use in their production plant.
    \item \textbf{Accessibility:}
    Preprint of submitted version: \url{}

    \noindent Source code: \url{}

    \noindent Curated dataset: \url{}
\end{itemize}

\begin{itemize}[wide, labelindent=0pt]
    \setlength\itemsep{-2pt}
    \item \textbf{Full reference:}
    Irastorza, M. and Ponce-Vanegas, F.
    (2026).
    The Tustin Method for approximating eigenvalues of delay systems.
    \textit{J. Comput. Appl. Math.},
    472, 116772.
    \url{https://doi.org/10.1016/j.cam.2025.116772}
    \item \textbf{Role and contribution:}
    Conceptualization, methodology, software, supervision, and ma\-nus\-cript writing.
    \item \textbf{Evidence of impact:}
    This is the bachelor's thesis in mathematics of Markel Irastorza.
    \item \textbf{Accessibility:}
    Preprint of submitted version:
    \url{http://hdl.handle.net/20.500.11824/1841}.
    
    \noindent Source code: \url{https://gitlab.bcamath.org/mirastorza/Semi-discretization_and_the_Tustin_method}
\end{itemize}

\begin{itemize}[wide, labelindent=0pt]
    \setlength\itemsep{-2pt}
    \item \textbf{Full reference:}
    Ponce-Vanegas, F., Bartfai, A., and Dombovari, Z.
    (2025).
    Semi-analytical Estimation for the Escape of Solutions of Linear Differential Equations with Slowly Varying Coefficients.
    \textit{SIAM J. Appl. Math.},
    85(4), 1519-1549.
    \url{https://doi.org/10.1137/24M1685481}
    \item \textbf{Role and contribution:}
    First author, conceptualization, methodology, software, and ma\-nus\-cript writing.
    \item \textbf{Evidence of impact:}
    \item \textbf{Accessibility:}
    Preprint of submitted version:
    \url{http://hdl.handle.net/20.500.11824/1830}.

    \noindent Source code: \url{https://gitlab.bcamath.org/fponce/variable-coefficients}
\end{itemize}

\begin{itemize}[wide, labelindent=0pt]
    \setlength\itemsep{-2pt}
    \item \textbf{Full reference:}
    Ponce-Vanegas, F., Bizzarri, M., and Barton, M.
    (2023).
    On and continuity of envelopes of rotational solids and its application to 5-axis CNC machining.
    \textit{CAGD},
    107, 102245.
    \url{https://doi.org/10.1016/j.cagd.2023.102245}
    \item \textbf{Role and contribution:}
    First author, conceptualization, methodology, software, and ma\-nus\-cript writing.
    \item \textbf{Evidence of impact:}
    \item \textbf{Accessibility:}
    Preprint of submitted version:
    \url{http://hdl.handle.net/20.500.11824/1638}

    \noindent Source code: \url{https://gitlab.bcamath.org/fponce/envelopes-solids-of-revolutions}
\end{itemize}

\noindent Please provide one or more permanent links where your publication record is available:
\begin{center}
    \renewcommand{\arraystretch}{1.2}
    \begin{tabular}{
        @{}|>{\raggedright}p{.25\textwidth-2\tabcolsep}
        |>{\raggedright\arraybackslash}p{.74\textwidth-2\tabcolsep}|
    @{}}
        \hline
        \textbf{Source} & \textbf{Link} \\ \hline
        ORCID & \url{https://orcid.org/0000-0002-1049-9752} \\ \hline
        Personal webpage & \url{https://a-wandering-mathematician.netlify.app/about} \\ \hline
    \end{tabular}
\end{center}

\end{document}