% !TeX program = xelatex

\documentclass[a4paper, 12pt]{article}

\usepackage{amsmath, mathtools}
\usepackage{blindtext}
\usepackage{fontspec}
\usepackage{layout}
\usepackage[
    verbose,
    paper=a4paper,
    % showframe,
    left=1in,
    right=1in,
    top=36pt,
    textheight=698pt,
    headheight=16pt,
    headsep=20pt, %top+headh+heads=72pt
    marginparwidth=0pt,
    marginparsep=0pt,
    includeheadfoot]{geometry}
\usepackage{fancyhdr}
\usepackage{lastpage}
\usepackage{titlesec}
\usepackage{epigraph}

\mathtoolsset{showonlyrefs}

% Set main font
\setmainfont{Times New Roman}

% Header and Footer
\pagestyle{fancyplain}
\fancyhf{}
\pagenumbering{arabic}
\fancyhead[C]{
    \textbf{Momentum MSCA Programme - Call 2}
}
\fancyfoot[R]{\scriptsize
    page \thepage \,out of\, \pageref{LastPage}}
\renewcommand{\headrulewidth}{0pt}
\renewcommand{\footrulewidth}{0pt}

\titleformat{\section}
    {\normalfont\bfseries}{\thesection.}{1em}{}

% Bibliography setup
\usepackage[style=ieee, backend=biber, maxnames=4]{biblatex}
\addbibresource{refs.bib}
\defbibheading{bibliography}{\section*{References}}

%%%%%%%%%%%%%%%%%%%%%%%%%%%%%%%%%%%%%%%%%%%
\begin{document}

\begin{center}
\fbox{\parbox{0.98\textwidth}{\centering
\textbf{Research proposal}
}}
\end{center}

\begin{center}
    \renewcommand{\arraystretch}{1.6}
    \begin{tabular}{|p{0.26\textwidth}|p{0.68\textwidth}|}
        \hline
        Name of the Applicant: & Felipe Eduardo Ponce Vanegas \\
        \hline
        Application Title: &  \\
        \hline
    \end{tabular}
\end{center}

\section{Excellence}

Reliable, autonomous, and smooth running machines is the most desirable state of a ma\-nu\-facturing plant.
To achieve, or approach, this utopian scenario it is essential to understand
the response of a machine to external forces so that
unstable operating conditions can be avoided or controlled.
In this proposal we aim at understanding the response of machine tools and robots in the context of Advance Manufacturing, in particular,
for processes that involve metal removal.

To understand the response of a machine --- an activity known as \emph{System Identification} ---
a basic component is modelling.
Unlike many successful data-driven models for image or language processing,
the amount of available data to train models in Advance Manufacturing is much more limited.
To compensate for the scarcity of data (or relevant information),
physically motivated models are still in active use in the field.

Bayesian statistics offers a rigorous answer to the question:
how much do the data support a model?
This is particularly important when decisions must be made with little relevant information.
Bayes's theorem states that
\begin{equation}
    p(M \mid D) \propto p(D \mid M)\,p(M),
\end{equation}
which means that the probability \(p(M \mid D)\) of a model \(M\) given the observed data or evidence \(D\) (the posterior) is
proportional to the probability \(p(D \mid M)\) of the data given the model (the likelihood) times
the probability \(p(M)\) of the model (the prior).
The posterior provides a tool to make predictions that also includes
the uncertainty in the selection of the model.

Physical models are usually of the form
\begin{equation}
    M\ddot{x} + C\dot{x} + Kx + G(t, x, \dot{x}) = f(t),
\end{equation}
where \(x\) is the displacement from equilibrium,
\(f\) is an external force or excitation,
\(M\), \(C\), and \(K\) are matrices representing the mass, damping, and stiffness, and
\(G\) represents additional linear or nonlinear effects.



\section{Impact}

Advance Manufacturing is a strategic sector for the Europe Union to
ensure the position of the region as an industrial, world leader \cite{industrialforum}.


\section{Implementation}


\printbibliography

\end{document}